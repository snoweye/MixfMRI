
{\color{red} \bf Warning:}
The findings and conclusions in this article have not been
formally disseminated by the U.S. Food and Drug Administration
and should not be construed to represent any determination or
policy of any University, Institution, Agency, Adminstration
and National Laboratory.

This document is written to explain the main
function of \pkg{MixfMRI}~\citep{Chen2018}, version 0.1-0.
Every effort will be made to ensure future versions are consistent with
these instructions, but features in later versions may not be explained
in this document.



\section[Introduction]{Introduction}
\label{sec:introduction}
\addcontentsline{toc}{section}{\thesection. Introduction}

The main purpose of this vignettes is to demonstrate basic usages of
\pkg{MixfMRI} which is prepared for a developed methodology,
simulation sutdies, data analyses in
``Improved Activation Detection in Single-Subject fMRI
Studies''~\citep{ChenMaitra2018}.
The methodology mainly utilizes model-based clustering (unsupervised)
of fMRI data that identifies regions of brain activation associated
with stimula.
The implemented methods includes
2D and 3D clustering analyses and segmentation analyses for
fMRI signals.
For simplification,
only 2D clustering is demonstrated in this vignettes.
In this package, the fMRI signals for brain activation,
e.g. response to stimulate of interesting, are represented
by p-values for every voxel.
The clustering and segmentation analyses
mainly identify active voxels/signals (in terms of small p-values)
from normal brain behaviors within a single subject.

Note that the p-values may be derived from statistical models where typically
a proper experimial design is required.
The methods and analyses are for exprotorary purpose only, so that no claims
nor statements about significance level of p-values associated with
active voxels/signals will be needed/made. 
However, the analyses allow to prespecify expected upper bounds
for the proportion of active voxels/signals that can ease burdens
of selecting unreasonable amounts of active vosels.

For large datasets,
the methods and analyses are also implemented in a distributed manner
especially using SPMD programming framework.
The package also includes workflows which utilize SPMD techniques.
The workflows serve as examples of data analyses and
large scale simulation studies.
Several workflows are also built in for automatically
process clusterings, hypotheses, merging clusters, and visualizations.
See Section~\ref{sec:workflows} and files in \code{MixfMRI/inst/workflow/}
for information.


\subsection[Dependent Packages]{Dependent Packages}
\addcontentsline{toc}{subsection}{\thesubsection. Dependent Packages}

The \pkg{MixfMRI} depends on other \proglang{R} packages to be best functioning
even though they are not required.
Some examples, functions and workflows of the
\pkg{MixfMRI} may need utilities of those dependent packages.
For instance,
\\
\hspace*{0.5cm}
{\bf Imports:} \pkg{MASS}, \pkg{Matrix}, \pkg{RColorBrewer},
\pkg{fftw}, \pkg{MixSim}, \pkg{EMCluster}.
\\
\hspace*{0.5cm}
{\bf Enhances:} \pkg{pbdMPI}, \pkg{AnalyzeFMRI}, \pkg{oro.nifti}.


\subsection[The Main Function]{The Main Function}
\addcontentsline{toc}{subsection}{\thesubsection. The Main Function}

The main function, \code{fclust()}, implements model-based clustering
using EM algorithm~\citep{McLachlan1996} for fMRI signal data and provides
unsupervised clustering results that identify active regions in brain.
The \code{fclust()} contains an initialization method and
EM algorithms for clustering fMRI signal data which have two parts:
\begin{itemize}
\item
\code{PV.gbd} for p-value of signals associated with voxels, and
\item
\code{X.gbd} for voxel information/locations in either 2D or 3D,
\end{itemize}
where \code{PV.gbd} is of length $N$ (number of voxels) and
\code{X.gbd} is of dimension $N \times 2$ or $N \times 3$ (for 2D or 3D).
Each signal (per voxel) is assumed to follow a mixture distribution
of \code{K} components with mixing proportion \code{ETA}.
Each component has
two independent coordinates (one for each part)
with density functions: Beta and multivariate
Normal distributions, for each part of fMRI signal data.
\\

{\bf Beta Density:}\\
The first component (\code{k = 1})
is restricted by \code{min.1st.prop} and $Beta(1, 1)$
distribution. The rest \code{k = 2, 3, ..., K - 1} components have different
$Beta(alpha, beta)$ distributions with \code{alpha < 1 < beta} for all
\code{k > 1} components.
This coordinate mainly represents the results of testing statistics
for determining activation of voxels (small p-values).
Note that the testing statistics may be developed/smoothed/computed
from a time course model associated with voxel behaviors.
See the main paper \citet{ChenMaitra2018} for information.
\\

{\bf Multivariate Normal Density:}\\
\code{model.X = "I"} is for identity covariance matrix of multivariate Normal
distribution, and \code{"V"} for unstructured covariance matrix.
\code{ignore.X = TRUE} is to ignore \code{X.gbd} and normal density,
i.e. only Beta density is used.
Note that this coordinate (for each axis)
is recommended to be normalized in $(0, 1)$ scale
which is the same scale of Beta density. No effects are expected from the
rescaling \code{X.gbd} from modeling perspective.
\\

In this package, the two parts,
\code{PV.gbd} and \code{X.gbd}, are assumed independent because the
dependence were typically ignored in the designed models
(time course of voxel activation associated with the stimula)
from which the \code{PV.gbd} are dervided.
Even though, the goal of the main function is to provide spatial
clusters (in addition to the \code{PV.gbd}) indicating spatial correlations.

Currently, APECMa~\citep{Chen2011}
and EM algorithms are implemented with EGM algorithm~\citep{Chen2013}
to speed up convergence when MPI and \pkg{pbdMPI}~\citep{Chen2012}
are available.
RndEM initialization~\citep{Maitra2009} is also implemented for obtaining
good initial values that may increase chance of convergence.


\subsection[Datasets]{Datasets}
\addcontentsline{toc}{subsection}{\thesubsection. Datasets}

The package has been built with several datasets including
\begin{itemize}
\item
three 2D phantoms, \code{shepp0fMRI}, \code{shepp1fMRI}, and \code{shepp2fMRI},
\item
one 3D voxels dataset, \code{pstats}, in p-values,
\item 
two small 2D voxels datasets, \code{pval.2d.complex} and \code{pval.2d.mag},
in p-values
\item
two toy examples, \code{toy1} and \code{toy2}.
\end{itemize}


\subsection[Examples]{Examples}
\addcontentsline{toc}{subsection}{\thesubsection. Examples}
The scripts in \code{MixfMRI/demo/} have several examples that demonstrate
the main function, the example datasets and other utilities in this package.
For quick start,
\begin{itemize}
\item
the scripts \code{MixfMRI/demo/fclust2d.r} and \code{MixfMRI/demo/fclust3d.r}
show the basic usage of the main function \code{fclust()} using
two toy datasets,
\item
the scripts \code{MixfMRI/demo/maitra_2d.r} and
\code{MixfMRI/demo/shepp.r} show and visualize
examples on how to generate simulated datasets with given overlap levels, and
\item
the scripts \code{MixfMRI/demo/alter_*.r} show alternative methods.
\end{itemize}


\subsection[Workflows]{Workflows}
\label{sec:workflows}
\addcontentsline{toc}{subsection}{\thesubsection. Workflows}
The package also have several workflows established for simulation studies.
The main examples are located in \code{MixfMRI/inst/workflow/simulation/}.
See the file \code{create_simu.txt} that generates scripts for simulations.

The files under \code{MixfMRI/inst/workflow/spmd/} have main scripts of
workflows.
Note that MPI and \pkg{pbdMPI} are required for workflows because these
simulations require potentially long computing time.

